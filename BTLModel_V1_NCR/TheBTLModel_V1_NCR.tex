
\documentclass{article}
\usepackage[utf8]{inputenc}
\usepackage{graphicx}
\usepackage{epstopdf}
\usepackage{amsmath}
\usepackage{xcolor}

\title{The BTL Model}
\author{Xiaolin Shen}

\begin{document}

\maketitle


\section{The BTL Model --}
The winning item $w\in W^d$ is a skyline object. To define the probability of skyline events, use the BTL model.

\begin{align*}
	p(i \succ j |u)=\frac{ui}{ui+ uj} \\
	p(i \prec j |u)=\frac{uj}{uj+ ui} \\
\end{align*}

If the favorite aspect in current session is $k$, then following the definition of skyline object, $w\in W^d$ is superior in the favorite aspect,  and is not worse than other items in other aspects. $w\in W^d$ ranks higher than item $v\in L^d$ in the aspect $k$ is
$$p^k(w \succ v)=\frac{u_k w_k}{u_k w_k+u_k v_k}$$
 the probability of item $w\in W^d$ ranks not lower than item $v\in L^d$ in the aspect $k'!=k$. Therefore,
 $$p^{k'}(w \succeq v)=1- p^{k'} (v \succ w) = 1-\frac{u_k v_k}{u_k w_k+u_k v_k}=\frac{u_k w_k}{u_k w_k+u_k v_k}$$.

 Suppose $g$ is a $K-$dim vector, with one and only one component to be equal to $1$, for each pair, $p(<w,v>|g_k=1) = p^k(w\succ v) \times \Pi_{k'\neq k} p^{k'}(w \succeq v)$. then the probability of generating a session observation $d$ given the hidden aspect $a$ is defined as:

\begin{align}\label{equ:skyline}
p(d|g,V,U)
		=\Pi_{w\in W^d, v\in L^d} \Pi_{k=1}^{K}[ {\frac{u_k w_k}{u_k w_k+u_k v_k}}^{g_k} { \frac{u_k w_k}{u_k w_k+u_k v_k}}^{1-g_k}]\\
=\Pi_{w\in W^d, v\in L^d} \Pi_{k=1}^{K}[ {\frac{u_k w_k}{u_k w_k+u_k v_k}}]
\end{align}


First, let's use  $\gamma(d,k,\Theta^t)$ to denote the conditional probability $p(g_k=1|d,\Theta^t)$ given parameters in the $t-$th round, when the current session specific favorite aspect is $g_k=1$, defined as follows
\begin{align}\label{equ:conditional}
\gamma(d,k,\Theta^t) &=\frac{p(d,g|\Theta^t)}{\Sigma_g p(d,g|\Theta^t)} = \frac{p(g|\Theta^t)p(d|\Theta^t,g)}{\Sigma_g p(g|\Theta^t)p(d|\Theta^t,g)}\\\nonumber
&=\frac{u_{g^k} \Pi_{w \in W^d, v\in L^d} \Pi_{k=1}^{K}\frac{u_k w_k}{u_k w_k+u_k v_k}}{\Sigma_{k=1}^K u_{g^k} \Pi_{w \in W^d, v\in L^d} \Pi_{k=1}^{K}\frac{u_k w_k}{u_k w_k+u_k v_k}}
\end{align}

Note that $\forall d, \Sigma_k \gamma(d,k,\Theta^t)=1$, $g^k$ denote u's favorite aspect.

\subsection{E-step}
 In the E-step of $t-$th  EM round, compute the expectation $Q(\Theta^t)=E_{G} \ln p(D,G|\Theta) $

\begin{align}\label{equ:estep}
E_{G} \ln p(D,G|\Theta) & =\Sigma_{d} \Sigma_{k=1}^K \gamma(d,k,\Theta^t) \ln p(d,g|\Theta)\\ \nonumber
& = \Sigma_d \Sigma_{k=1}^K \gamma(d,k,\Theta^t) \{ \ln u_{g^k}  \\ \nonumber
&+ \Sigma_{w\in W_d, v\in V_d}[\ln \frac{u_k w_k}{u_k w_k+u_k v_k}] \}
\end{align}

\subsection{M-step}
\subsubsection{For u}
firstly, maximize $Q(\Theta^t)$ with respect to $U$. For each $u \in U$, eliminating constant terms, we have:

By setting the partial derivative of $\frac{\partial \tilde{Q}(u_k,\Theta^t)}{\partial u_k}=0$, we have:
\begin{align}
\Sigma_{d} \Sigma_{k=1}^K \Sigma_{w\in W_d, v\in V_d}(\frac{1}{u_k}+\frac{ w_k}{u_k w_k}-\frac{v_k+w_k}{u_k w_k+u_k v_k}) = 0
\end{align}

\begin{equation}
u_k =\Sigma_{u(d)=u}\gamma(d,k,\Theta^t)
\end{equation}

\subsubsection{For v}
secondly, maximize $Q(\Theta^t)$ with respect to $V$.
for the :
\begin{equation*}
\ln \frac{y}{x} \geq 1- \frac{x}{y}
\end{equation*}\\
we can derive a lower bound for the log-likelihood over the complete data, given the parameters learnt from previous round.
hence, we obtain a minorization function of $\tilde{Q}(\Theta^t)$.\\
\begin{equation}
\begin{aligned}
\tilde{Q}(\Theta^t) &=\Sigma_d \Sigma_k \gamma(d,k,\Theta^t) \Sigma_{w\in W_d, v\in L_d}  \{[\ln (u_k w_k) + 1 - \ln (u_k^t w_k^t+u_k^t v_k^t) - \frac{u_k w_k+u_k v_k}{u_k^t w_k^t+u_k^t v_k^t}]\}
\end{aligned}
\end{equation}


$\tilde{Q}(\Theta^t)$ can be seperated for each item $v$. Considering only the $k-$th component $v_k$, $\tilde{Q}(v_k,\Theta^t)$ involves two terms, one of which is relevant to observations $d\in W(v)$ where $v$ acts as skyline object, the other is relevant to observations $d \in L(v)$ where $v$ acts as comparisons, $\tilde{Q}(v_k,\Theta^t)=\tilde{Q}^1(v_k,\Theta^t)+\tilde{Q}^2(v_k,\Theta^t)$. Removing all constants and irrelevant terms for $v_k$, we have the following minorizing function:
\begin{equation}
\begin{aligned}
\tilde{Q}^1(v_k,\Theta^t) & =\Sigma_{d\in W(v)} |L_d| \ln u_k v_k -v_k\Sigma_{d\in W(v)}\Sigma_{v'\in L_d} [\frac{u_k \gamma(d,k,\Theta^t)}{ \alpha(u,w,v,k)}]\\ \nonumber
\tilde{Q}^2(v_k,\Theta^t) & =-v_k \Sigma_{d\in L(v)}\Sigma_{v'\in W_d} [\frac{u_k \gamma(d,k,\Theta^t)}{\alpha(u,w,v,k)}]
\end{aligned}
\end{equation}
 where:\\

 $|L_d|$ is the number of objects being dominanted in $d$,
 \\ $\alpha(u,w,v,k)=u_k^t w_k^t+u_k^t v_k^t$.
 \\

By setting the partial derivative of $\frac{\partial \tilde{Q}(v_k,\Theta^t)}{\partial v_k}=0$, we have:


\begin{equation}
\begin{aligned}
\frac{1}{v_k}= &\frac{\Sigma_{d\in W(v)}\Sigma_{v'\in L_d} [\frac{u_k \gamma(d,k,\Theta^t)}{ \alpha(u,w,v,k)}]}{\Sigma_{d\in W(v)}|L_d|}\\\nonumber
 & + \frac{\Sigma_{d\in L(v)}\Sigma_{v'\in W_d} [\frac{u_k \gamma(d,k,\Theta^t)}{\alpha(u,w,v,k)}] }{\Sigma_{d\in W(v)}|L_d|}
\end{aligned}
\end{equation}\\

\end{document}
