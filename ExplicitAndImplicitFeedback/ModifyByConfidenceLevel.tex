
\documentclass{article}
\usepackage[utf8]{inputenc}
\usepackage{graphicx}
\usepackage{epstopdf}
\usepackage{amsmath}

\title{Modify By Confidence Level}
\author{Xiaolin Shen}

\begin{document} 

\maketitle


\section{The BTL Model}

\subsection{E-step}
\begin{eqnarray}
p(g|d,\Theta^{t}) & \propto   p(g|u,\Theta^{t}) p(d|g,\Theta^{t}) \\ \nonumber
 p(g_k=1|d,\Theta^{t}) & \propto  u_k^{t} \frac{w_{k}^t} {w_{k}^t+\theta^t l_{k}^t} \prod_{k'\neq k}  [\frac{\theta^t w_{k'}^t} {l_{k'}^t + \theta^t w_{k'}^t}]
\end{eqnarray}

First, let's use $\gamma(d,k,\Theta^t)$ to denote the conditional probability $p(g_k=1|d,\Theta^t)$ given parameters in the $t-$th round, when the current session specific favorite aspect is $g_k=1$, defined as follows

\begin{align}\label{equ:conditional}
\gamma(d,k,\Theta^t) &=\frac{p(d,g|\Theta^t)}{\Sigma_g p(d,g|\Theta^t)} = \frac{p(g|\Theta^t)p(d|\Theta^t,g)}{\Sigma_g p(g|\Theta^t)p(d|\Theta^t,g)}\\\nonumber
&=\frac{u_k \Pi_{w \in W^d, v\in L^d} \frac{w_k}{w_k+\theta^t v_k}\Pi_{k'\neq k}\frac{\theta^t w_{k'}}{v_{k'}+\theta^t w_{k'}}}{\Sigma_{k=1}^K u_k \Pi_{w \in W^d, v\in L^d} \frac{w_k}{w_k+\theta^t v_k}\Pi_{k'\neq k}\frac{\theta^t w_{k'}}{v_{k'}+\theta^t w_{k'}}}
\end{align}

Note that $\forall d, \Sigma_k \gamma(d,k,\Theta^t)=1$.



\subsection{S-step}
then simply add an S-step after the E-step,the value of $g$ for each session $d$ is
 \begin{equation}
 k \sim u_k^{t} \frac{w_{k}^t} {w_{k}^t+\theta^t l_{k}^t} \prod_{k'\neq k}  [\frac{\theta^t w_{k'}^t} {l_{k'}^t + \theta^t w_{k'}^t}]
 \end{equation}

 In the E-step of $t-$th  EM round, compute the expectation $Q(\Theta^t)=E_{G} \ln p(D,G|\Theta) $

\begin{equation}
\begin{aligned}
E_{G} \ln p(D,G|\Theta)  &= \Sigma_d \gamma(d,k,\Theta^t)\ln p(d,g|\Theta)\\\nonumber
& = \Sigma_d \gamma(d,k,\Theta^t) \{ \ln u_k + \Sigma_{w\in W_d, v\in V_d} [\ln \frac{w_k}{w_k +\theta v_k} +\Sigma_{k'\neq k} \ln \frac{\theta w_{k'}}{v_{k'}+\theta w_{k'}}]\}
\end{aligned}
\end{equation}
\\
\subsection{M-step}

\subsubsection{For u}
first maximize $Q(\Theta^t)$ with respect to $U$. For each $u \in U$, eliminating constant terms, we have:

\begin{align}\label{equ:Lu}
\min -\Sigma_{u(d)=u} \Sigma_{k=1}^K \gamma(d,k,\Theta^t) \ln u_k\\ \nonumber
w.r.t \Sigma_k u_k =1
\end{align}

Solving the above Lagrange function Equ.~\ref{equ:Lu}, we get

\begin{equation}\label{equ:u}
u_k =\frac{\Sigma_{u(d)=u}\gamma(d,k,\Theta^t)}{\Sigma_{s=1}^K \Sigma_{u(d)=u}\gamma(d,s,\Theta^t)}
\end{equation}

\subsubsection{For v}
for the :
\begin{equation*}
\ln \frac{y}{x} \geq 1- \frac{x}{y}
\end{equation*}\\
we can derive a lower bound for the log-likelihood over the complete data, given the parameters learnt from previous round.
hence, we obtain a minorization function of $\tilde{Q}(\Theta^t)$.\\
\begin{equation}
\begin{aligned}
\tilde{Q}(\Theta^t) &= \Sigma_d \Sigma_k \gamma(d,k,\Theta^t) \Sigma_{w\in W_d, v\in L_d}  \{ [\ln w_k + 1 - \ln (w_k^t + \theta^t v_k^t) - \frac{w_k+\theta v_k}{w_k^t + \theta^tv_k^t}]+\\\nonumber
&\Sigma_{k'\neq k} [\ln (\theta w_{k'}) + 1 - \ln (v_{k'}^t + \theta^t w_{k'}^t) -  \frac{v_{k'}+\theta w_{k'}}{v_{k'}^t + \theta^t w_{k'}^t}]
\}
\end{aligned}
\end{equation}


 $\tilde{Q}(\Theta^t)$ can be seperated for each item $v$. Considering only the $k-$th component $v_k$, $\tilde{Q}(v_k,\Theta^t)$ involves two terms, one of which is relevant to observations $d\in W(v)$ where $v$ acts as skyline object, the other is relevant to observations $d \in L(v)$ where $v$ acts as comparisons, $\tilde{Q}(v_k,\Theta^t)=\tilde{Q}^1(v_k,\Theta^t)+\tilde{Q}^2(v_k,\Theta^t)$. Removing all constants and irrelevant terms for $v_k$, we have the following minorizing function:


\begin{equation}
\begin{aligned}
\tilde{Q}^1(v_k,\Theta^t) & = \Sigma_{d\in W(v)} |L_d| \ln v_k -v_k\Sigma_{d\in W(v)}\Sigma_{v'\in L_d} [\frac{\gamma(d,k,\Theta^t)}{ \alpha(v,v',k,\Theta^t)} +\Sigma_{k'\neq k}\frac{\theta^t\gamma(d,k',\Theta^t)}{\alpha(v',v,k,\Theta^t)}]\\ \nonumber
\tilde{Q}^2(v_k,\Theta^t) & = -v_k \Sigma_{d\in L(v)}\Sigma_{v'\in W_d} [\frac{\theta^t \gamma(d,k,\Theta^t)}{\alpha(v',v,k,\Theta^t)}+\Sigma_{k'\neq k} \frac{\gamma(d,k',\Theta^t)}{\alpha(v,v',k,\Theta^t)}]
\end{aligned}
\end{equation}
 where:\\

 $|L_d|$ is the number of objects being dominanted in $d$,
 \\ $\alpha(v,v',k,\Theta^t)=v_k^t + \theta^t {v'}_k^t$.
 \\
By setting the partial derivative of $\frac{\partial \tilde{Q}(v_k,\Theta^t)}{\partial v_k}=0$, we have:



\begin{equation}
\begin{aligned}
\frac{1}{v_k}= &\frac{\Sigma_{d\in W(v)}\Sigma_{v'\in L_d} [\frac{\gamma(d,k,\Theta^t)}{ \alpha(v,v',k,\Theta^t)} +\Sigma_{k'\neq k}\frac{\theta^t\gamma(d,k',\Theta^t)}{\alpha(v',v,k,\Theta^t)}]}{\Sigma_{d\in W(v)}|L_d|}\\\nonumber
 & + \frac{\Sigma_{d\in L(v)}\Sigma_{v'\in W_d} [\frac{\theta^t \gamma(d,k,\Theta^t)}{\alpha(v',v,k,\Theta^t)}+\Sigma_{k'\neq k} \frac{\gamma(d,k',\Theta^t)}{\alpha(v,v',k,\Theta^t)}] }{\Sigma_{d\in W(v)}|L_d|}
\end{aligned}
\end{equation}\\

More concrete:\\
if $\gamma(d,k,\Theta^t)$'s k equals to $v_k $'s k:
\begin{equation*}
\begin{aligned}
\frac{1}{v_k}= &\frac{\Sigma_{d\in W(v)}\Sigma_{v'\in L_d} [\frac{\gamma(d,k,\Theta^t)}{ \alpha(v,v',k,\Theta^t)} ]}{\Sigma_{d\in W(v)}|L_d|}+ \frac{\Sigma_{d\in L(v)}\Sigma_{v'\in W_d} [\frac{\theta^t \gamma(d,k,\Theta^t)}{\alpha(v',v,k,\Theta^t)}] }{\Sigma_{d\in W(v)}|L_d|}
\end{aligned}
\end{equation*}\\
else:
\begin{equation*}
\begin{aligned}
\frac{1}{v_k}= &\frac{\Sigma_{d\in W(v)}\Sigma_{v'\in L_d} [\frac{\theta^t\gamma(d,k',\Theta^t)}{\alpha(v',v,k,\Theta^t)}|_{k'=k}]}{\Sigma_{d\in W(v)}|L_d|}+\frac{\Sigma_{d\in L(v)}\Sigma_{v'\in W_d} [\frac{\gamma(d,k',\Theta^t)}{\alpha(v,v',k,\Theta^t)}|_{k'=k}] }{\Sigma_{d\in W(v)}|L_d|}
\end{aligned}
\end{equation*}\\

\subsubsection{For $\theta$}
Fix $u,v$, update $\theta$ by:

  Fix $v \in V$ and $u \in U$, rearranging $Equ.9$, we have the solution for $\frac{\partial \tilde{Q}(\Theta^t)}{\partial \theta}=0$ as:

 \begin{equation}
\theta = \frac{(K-1)\Sigma_d |W_d| |L_d|}{\Sigma_d \Sigma_k \gamma(d,k,\Theta^t) \Sigma_{w,v} [\frac{v_k}{\alpha(w,v,k,\Theta^t)}+\Sigma_{k'\neq k} \frac{w_{k'}}{\alpha(v,w,k',\Theta^t)}]}
 \end{equation}
Note that :\\
in the $\Sigma_d \Sigma_k \gamma(d,k,\Theta^t)$ only summing these d which $\gamma(d,k,\Theta^t)$'s k equals to d's k.
\end{document}
